\chapter{Introduction} \label{chap:chap-1}

% if you want a short header you can use the following command
% \chapter[short-header-name]{chapter-title} \label{chap:chap-1}


%% add your chapter text here
%\blindtext \cite{dirac}
\epigraph{Fiat Lux\footnotemark[1]} % Adds a footnote marker to the epigraph text

\footnotetext[1]{And Go Bears} % Defines the text of the footnote



\section{Background and Motivation}

Navigating the changing landscape of Computer Science is more difficult than ever, for both newcomers to the field and seasoned professionals aiming to stay informed and up-to-date on trends. With the increasing value of technical skills, Computer Science (CS) is increasingly interfacing with and transforming numerous industries, from healthcare to finance, and even the humanities. This shift has led to an increased demand for CS knowledge, as individuals across various fields recognize the value of acquiring technical expertise. While some individuals may explore university degree programs to understand common coursework and prerequisites, gaining access to these classes can be challenging due to high costs, rigid schedules, or other limitations.

Open-access courseware (OCW) has emerged as a valuable alternative, offering high-quality educational content for free or at a significantly reduced cost. These resources make it possible for self-learners to access university-level courses and specialized content from leading institutions, democratizing education and resources for those with limited academic opportunities. Platforms like Coursera, edX, and MIT OpenCourseWare have made it easier than ever to learn technical skills from anywhere in the world.

However, self-learners that turn to open-access courseware face an overwhelming number of courses and learning platforms. Popular platforms like Coursera and edX provide a wealth of content, but with it comes an unintended side effect: choice overload. This abundance of options can make it difficult for learners to find courses that align with their backgrounds, goals, and existing skills. This can lead to their enrolling in overly general introductory courses and ending up with a generic background, ultimately dampening their unique interests, perspectives, and potential contributions to the field.

Moreover, the complexity of selecting the right learning pathway is enhanced by the challenge of understanding prerequisite requirements. Learners often lack clear guidance on which courses align with their current skill levels or goals, leading them to enroll in overly generic or irrelevant courses. This results in learners accumulating knowledge without a coherent learning pathway, ultimately hindering their learning success and overall satisfaction.


\section{Problem Statement}

The current landscape of online learning platforms leaves self-learners with two primary challenges: choosing the right courses amidst the overwhelming number of options and ensuring that their chosen courses align with their background and future goals.  Users cannot find the course(s) suited to their needs using existing learning platforms. Existing course recommendation systems on open courseware platforms rely heavily on keyword-based search and popularity metrics, leading to generic recommendations that often prioritize widely-enrolled or popular courses. This lack of personalized guidance not only hinders learning success and satisfaction but also limits the effective use of their diverse backgrounds.

This gap highlights the need for a more advanced solution that not only recommends courses but also tailors learning pathways to each learner’s specific needs, accounting for their prior knowledge, prerequisites, and goals.

\section{Research Objectives}

We aim to solve this problem by developing Gaita, a Retrieval Augmented Generation (RAG) chatbot that generates personalized learning pathways in Computer Science. Gaita leverages a database of over 1,200 open-access Computer Science courses from Coursera and MIT OpenCourseWare to provide tailored recommendations based on learners’ specific backgrounds, needs, and ambitions. 

As AI becomes increasingly integrated into every field, it is critical to equip individuals with the technical tools and knowledge to enhance their existing roles and passions, rather than focusing solely on technical professions. We aim to change how AI education can be harnessed to empower people in their diverse pursuits and provide personalized pathways that align with learners’ unique goals and interests.
\chapter{Introduction} \label{chap:chap-1}

% if you want a short header you can use the following command
% \chapter[short-header-name]{chapter-title} \label{chap:chap-1}


%% add your chapter text here
%\blindtext \cite{dirac}

\section{Background and Motivation}

Navigating the changing landscape of Computer Science is more difficult than ever, for both newcomers to the field and seasoned professionals aiming to stay informed and up-to-date on trends. With the increasing value of technical skills, Computer Science (CS) is increasingly interfacing with and transforming numerous industries, from healthcare to finance, and even the humanities. This shift has led to an increased demand for CS knowledge, as individuals across various fields recognize the value of acquiring technical expertise. While some individuals may explore university degree programs to understand common coursework and prerequisites, gaining access to these classes can be challenging due to high costs, rigid schedules, or other limitations.

Open-access courseware has emerged as a valuable alternative, offering high-quality educational content for free or at a significantly reduced cost. These resources make it possible for self-learners to access university-level courses and specialized content from leading institutions, democratizing education and resources for those with limited academic opportunities. Platforms like Coursera, edX, and MIT OpenCourseWare have made it easier than ever to learn technical skills from anywhere in the world.

However, self-learners that turn to open-access courseware face an overwhelming number of courses and learning platforms. Popular platforms like Coursera and edX provide a wealth of content, but with it comes an unintended side effect: choice overload. This abundance of options can make it difficult for learners to find courses that align with their backgrounds, goals, and existing skills. This can lead to their enrolling in overly general introductory courses and ending up with a generic background, ultimately dampening their unique interests, perspectives, and potential contributions to the field.

Moreover, the complexity of selecting the right learning pathway is enhanced by the challenge of understanding prerequisite requirements. Learners often lack clear guidance on which courses align with their current skill levels or goals, leading them to enroll in overly generic or irrelevant courses. This results in learners accumulating knowledge without a coherent learning pathway, ultimately hindering their learning success and overall satisfaction.


\section{Problem Statement}

\section{Research Objectives}
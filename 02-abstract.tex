\chap{Abstract} 


%%%% your abstract goes here (word limit: 350)


Navigating the changing landscape of Computer Science is more difficult than ever, for both newcomers to the field and seasoned professionals aiming to stay informed and up-to-date on trends.

With a plethora of online learning resources, self-learners are often left to haphazardly browse the web and enroll in the first or most affordable program they find—a consequence of choice overload that not only impedes their success and overall satisfaction, but also neglects to consider their unique perspectives and potential contributions to the field. It’s extremely difficult to figure out which course(s) are right for you, and even harder to keep your goals in mind when worrying about prerequisites you might need. This can lead to enrolling in overly general introductory courses and ending up with a generic background in the field, ultimately making it harder to stand out in a competitive job market or make impactful contributions in your field.

We aim to solve this problem by developing Gaita, a Retrieval Augmented Generation (RAG) ChatBot that generates personalized learning pathways from our database of over 1,200 open-access Computer Science courses from Coursera and MIT OpenCourseWare tailored to individuals’ specific backgrounds, needs, and ambitions.


%% list of keywords seperated by comma
\keywords{Computer Science Education, Personalized Education, Recommender Systems, Retrieval Augmented Generation}


%%%%  committee members (add it right after the abstract w/o page break)
\begin{singlespace}

    %% if you have co-advisor, add here w/ \vspace{0.1in} as shown below
    %% alternatively you can use minipage environment to put side-by-side
    \section*{Primary reader and thesis advisor}
    
    Dr. David Yarowsky \\
    Professor\\
    Department of Computer Science\\
    Johns Hopkins University, Baltimore MD 

% TODO: double check
    %\section*{Secondary reader}
    
    %Dr. Joshua Reiter\\
    %Professor\\
    %Department of Computer Science \\
    %Johns Hopkins University, Baltimore, MD 
    
    %\vspace{0.1in}
     

    %% you can add more readers if you have them on your committee 
    %% use \vspace{0.1in} in between members for clarity
    %% you can also place committee members side-by-side using `minipage`


\end{singlespace}
\chapter{Literature Review} \label{chap:chap-2}

%%%% OPTIONAL EPIGRAPH EXAMPLE
%\epigraph{Do not believe everything you see on the internet.}{-- Albert Einstein}


%%%% MUST: if the chapter is a reprint or submitted paper, you must declare it
%% you can use enumerate or itemize environment if you have more than one paper 
%% \mybibexclude{} will exclude this citation from the final bibliography
%% if this paper appears somewhere else then remove \mybibexclude{} command

%\begin{singlespace}         % you can also use `onehalfspace` to relax the spacing
%    This chapter is adapted from the following article with permission from <publisher>
    
%    \fullcite{einstein}. \mybibexclude{einstein}
%\end{singlespace} 


%%%% remove the following and add your chapter text here
\section{Introduction}
The growing demand for technical skills has led many learners to open-access courseware platforms like Coursera and edX. While these platforms offer a wealth of quality educational content, they also present challenges, such as navigating a vast selection of courses, identifying those that align with individual goals, and understanding prerequisite requirements. 

This leads to choice overload, where too many options and insufficient guidance result in frustration and poor learning outcomes. Existing recommendation systems rely on vanilla algorithms that fail to capture the unique needs and diverse backgrounds of learners. Recent advancements in AI, particularly through Large Language Models (LLMs) and Retrieval-Augmented Generation (RAG), offer new opportunities to address these challenges. By providing more dynamic and context-aware recommendations, these models can create personalized experiences. 

This chapter explores the obstacles faced by self-learners, the limitations of existing recommendation systems, and how LLMs can reshape personalized learning. It also identifies the key features of an effective system and the remaining gaps in current solutions.

\section{AI is Changing Every Field} 

Advancements in Artificial Intelligence (AI) are transforming a wide range of industries \cite{noauthor_ai_2023} and driving the demand for technical skills and Computer Science knowledge \cite{noauthor_reskilling_nodate}. From healthcare \cite{saxena_ai_2001} to education \cite{luckin_intelligence_2016} and creative industries \cite{noauthor_pdf_nodate}, AI is enabling new innovations that require expertise in areas such as Data Science, Machine Learning, and Software Development. In sectors like healthcare, AI is being used to analyze large datasets, improve diagnostic accuracy \cite{lenharo_ai_2023}, and enhance patient care \cite{khorsand_ai_2024}. Similarly, in finance, AI-powered algorithms are automating fraud detection and risk assessment \cite{noauthor_pdf_nodate}. These developments demonstrate the expanding influence of AI and highlight the increasing need for professionals with a strong technical foundation.

\subsection{There is a growing demand for technical skills}

As AI and technology continues to shape various fields, the demand for individuals with Computer Science knowledge has surged. According to recent studies, job postings requiring computer science and AI-related skills have increased significantly, emphasizing the rising value of these competencies in the job market \cite{noauthor_computer_nodate} \cite{noauthor_future_nodate}. 

More professionals and learners are recognizing the value of technical expertise, both to remain competitive in their current roles and to explore new career paths \cite{abe_future_2021}. Moreover, a growing number of students and early-career professionals are seeking knowledge in these areas \cite{noauthor_how_nodate}. This influx of learners from diverse backgrounds creates new challenges in identifying educational resources that are both relevant and appropriate for their various needs.

\subsection{Traditional education is not enough }
However, traditional educational programs are often not well-suited to meet this growing demand \cite{noauthor_opinion_nodate}. While university degrees offer comprehensive coursework, they are typically expensive and inflexible, making them inaccessible to many learners. These programs also struggle to meet the needs of continuing learners—those already in the workforce who wish to update their skills \cite{noauthor_lifelong_nodate}. High tuition costs and rigid schedules limit access to these traditional pathways, leaving many learners seeking alternative ways to gain the necessary expertise \cite{noauthor_college_nodate}.

\subsection{MOOCs and their limitations}
For these reasons, an increasing number of learners are turning to more flexible and affordable options, such as open-access courseware and Massive Open Online Courses (MOOCs) \cite{harish_online_2013} \cite{pampouri_massive_2021}. The rise of Massive Open Online Courses marked a significant shift in the accessibility of education. MOOCs were initially celebrated for their potential to democratize education, offering a plethora of high-quality courses from leading universities to anyone with internet connection. 

In 2009, a meta-analysis by the U.S. Department of Education examined evidence-based practices in online learning and found that students who received some or all of their education online performed better on average than students receiving exclusively in-person instruction \cite{means_evaluation_2009}. However, as the popularity of these platforms grew, studies revealed patterns of enrollment and attrition that highlighted several challenges. A preliminary case study on Australian open-access online education found that while MOOCs attracted a large and diverse group of enrollees, many learners struggled to stay engaged \cite{greenland_patterns_2014}. High attrition rates and low completion rates became characteristic of MOOCs, both emphasizing their capability for increasing access to education and raising concerns about their effectiveness in achieving long-term learning outcomes \cite{clow_moocs_2013}. 

\subsubsection{Low retention rates}

High attrition and low completion rates in Massive Open Online Courses (MOOCs) are influenced by several factors, including learners enrolling in courses that may not align with their backgrounds or goals. This misalignment can stem from the overwhelming number of available courses, leading to choice overload and suboptimal decision-making. A study by Aldowah et al. identified key factors contributing to student dropout in MOOCs, emphasizing students’ misalignment of prior experience, difficulty, and academic skills and abilities \cite{noauthor_factors_nodate}. These elements highlight the importance of aligning course selection with individual learner profiles to enhance retention. 

\subsubsection{Choice overload}

The abundance of learning resources provided by MOOCs presents a double-edged sword: while it offers learners a wide variety of courses suited to different backgrounds and interests, it also overwhelms individuals with options that may not suit their specific needs or goals. Choice overload occurs when individuals are presented with so many options that the decision-making process becomes difficult, leading to potential dissatisfaction, indecision, or regret  \cite{chernev_choice_2015}. This research by Chernev et al. identifies choice set complexity, decision task difficulty, preference uncertainty, and the decision goals as key factors that contribute to this phenomenon. Moreover, the authors find that choice overload can be used interchangeably with satisfaction/confidence, regret, choice deferral, and switching likelihood, which are equally powerful measures of choice overload. Moreover, choice overload results in deferral, frequent switching, and decreased satisfaction. 

These effects are highly relevant to the issues of low completion and high attrition rates in MOOCs. When faced with overwhelming course options, learners may struggle to choose courses that align with their goals and abilities, often resulting in enrollment in courses that are not well-suited to their needs. This misalignment can increase the cognitive load on learners, making it harder to stay engaged and committed to a course over time. Additionally, choice overload may lead learners to defer or abandon their course selections altogether, contributing to high dropout rates. Frequent switching between courses, driven by dissatisfaction or regret, disrupts the learning process and can further hinder completion.

\subsection{Recommender systems} 

Recommendation systems emerged in the 1990s to combat this very problem \cite{goldberg_using_1992}. They were initially developed as a response to the rapidly growing amount of information available on the internet. With an overwhelming number of media and online resources that the internet provided, users struggled to find relevant information among the vast options presented to them. Early recommendation systems were developed to help users navigate this abundance by filtering content based on user preferences, browsing behaviors, and prior interactions \cite{lutz_mafiaactive_1990}. These systems leveraged collaborative filtering and content-based algorithms to make personalized suggestions, thus reducing cognitive load and enhancing user experience. By recommending items most relevant to users, these systems became crucial tools for managing information overload and have since evolved to support decision-making in a wide variety of domains, from movie \cite{dong_brief_2022} to news recommendations \cite{resnick_grouplens_1994}.

\subsubsection{Recommender systems for MOOCs} 

While recommender systems have become ubiquitous in many fields, their application in education, particularly in MOOCs, is a work in progress. Existing course recommendation systems lack true personalization, and rely instead on simple metrics like content-based or collaborative filtering \cite{da_silva_systematic_2023} \cite{khalid_recommender_2020}. This approach fails to capture the diversity seen in the learners, and do not account for the unique backgrounds, skills, and learning goals of individuals, leading to a misalignment between the courses recommended, and the intent and expertise of the learner \cite{noauthor_reinforced_nodate}.

\subsubsection{Students come from varying backgrounds}

These systems undermine the diversity of learning experiences and perspectives that are essential for impactful contributions to both learners' fields and to the broader field of Computer Science. Pushing widely-enrolled, trending courses to all learners, regardless of their goals or backgrounds, leads to users enrolling in overly general courses and ending up with a generic background. This, in turn, dampens individuals’ diverse and unique interests and perspectives \cite{noauthor_2022_nodate}. Moreover, the absence of personalized support and guidance not only limits the diversity of learning outcomes but also amplifies inequities, disadvantaging learners with untraditional backgrounds and goals who lack the support needed to navigate their unique educational paths effectively \cite{dumont_promise_2023}.

The lack of personalization is especially problematic in the educational context, where relevance and appropriate alignment with the learner's background are crucial for successful educational outcomes \cite{noauthor_building_2017}. Without tailored recommendations, learners struggle to find courses that are well-suited to their knowledge level, leading to frustration, disengagement, and ultimately, high attrition and low completion rates in MOOCs.

\subsubsection{Bridging students' knowledge gap}

A major challenge for learners using MOOCs, especially those from untraditional backgrounds, is bridging their knowledge gap. Many learners enter these platforms with varying degrees of expertise, creating a need for personalized guidance on acquiring prerequisite knowledge. Without such guidance, learners often enroll in courses beyond their current skill levels and miss foundational knowledge crucial for success in advanced topics. Current recommendation systems on MOOC platforms frequently overlook this essential step, leading to frustration, high dropout rates, and unmet learning goals.

To address this challenge, some recent studies have explored recommendation systems that include prerequisite support to guide learners in their educational journeys. Parameswaran et al. (2009) highlight the importance of prerequisite recommendations, emphasizing that recommending courses based on a learner’s prior knowledge can greatly enhance learning outcomes \cite{parameswaran_recommendations_2009}. More recent studies propose implementing systems for assessing a learner’s current knowledge level and recommending courses that build on students’ prior knowledge to acquire more advanced skills \cite{noauthor_building_2017}. Soto et al. (2020) create a system to fill knowledge gaps in MOOCs by recommending parts of courses to students \cite{noauthor_vista_nodate}, and Wu et al. tackle this problem by proposing a Knowledge-Aware Meta-Concept that uses knowledge graphs for course recommendations \cite{wu_meta_2024}. Additionally, Gong et al. (2022) propose a novel approach called HinCRec-RL, which leverages reinforcement learning and network-based relationships between courses and knowledge concepts to provide more nuanced guidance to help learners fill specific knowledge gaps \cite{noauthor_reinforced_nodate}. By aligning course recommendations with each learner’s specific gaps and strengths, these systems can not only improve engagement and completion rates but also make it easier for learners to confidently advance to higher-level topics.

These challenges highlight the growing need for advanced recommendation systems capable of offering personalized pathways tailored to individual backgrounds and the evolving demands of various industries. For instance, learners from different fields—such as healthcare professionals aiming to learn data science or artists exploring computational creativity—require recommendations that recognize and address their specific contexts. Additionally, learners interested in data science applications may not need an in-depth knowledge of low-level programming languages, but instead will benefit more from foundational skills in data analysis, statistics, and introductory machine learning concepts. Without targeted guidance, learners may end up enrolling in irrelevant courses that do not align with their goals, which hinders both learner success and satisfaction when students struggle to progress meaningfully toward their objectives. In the context of MOOCs, this issue is further amplified: while large numbers of learners enroll in courses, only a small fraction complete them, often due to frustration with finding suitable courses and the absence of a structured pathway to support their learning journey


\subsection{LLMs are changing recommender systems}

Current MOOC platforms often struggle to provide personalized learning experiences that align with each learner's unique goals and existing knowledge base. Traditional recommendation systems typically rely on popularity metrics or keyword matching, and do not effectively capture individual learner preferences. To address this issue, there has been a shift towards more sophisticated learning systems that can bridge the knowledge gap and adapt to each learner’s background. Recent advancements in artificial intelligence, particularly through Large Language Models (LLMs) and Retrieval-Augmented Generation (RAG), offer new opportunities to address these challenges. By providing more dynamic and context-aware recommendations, these technologies have the potential to empower learners to pursue pathways that align with their interests and strengths and enhance the overall impact of online education \cite{khalid_recommender_2020}. 

LLMs, such as GPT-3, have demonstrated impressive general-purpose task-solving abilities, including the potential to approach recommendation tasks \cite{noauthor_language_nodate}. Unlike traditional search and recommender systems that rely on popularity metrics and keyword matching, LLMs can analyze natural language prompts and engage in conversational interactions to better understand learners’ specific goals, prior knowledge, and preferred learning styles. This allows for a deeper, more nuanced understanding of user needs. For instance, Zhang et al. (2023) propose a deep learning-based course recommendation method for an online education platform that leverages word embeddings to better align recommendations with each individual’s needs \cite{zhang_personalized_2023}. 

Retrieval Augmented Generation (RAG), originally developed for reducing hallucinations and enhancing language model performance on domain-specific tasks, show significant promise for recommender systems. It integrates LLMs with external data sources: an approach that allows for the retrieval of contextually relevant information, which can then used to generate personalized recommendations. By combining the strengths of retrieval-based methods and generative models, RAG offers more up-to-date and dynamic recommendations, aligning educational content with each learner's unique goals and existing knowledge base. Rao et al. (2024) leverage RAG in their RAMO system to enhance MOOC recommendations, addressing the "cold start" problem by providing tailored course suggestions through a conversational interface \cite{rao_ramo_2024}.
\section{What a System/Solution Should Include}

Synthesize everything 
- personalized and goal oriented
- take backgrounds into consideration + prerequisite stuff 

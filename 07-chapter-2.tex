\chapter{Literature Review} \label{chap:chap-2}

%%%% OPTIONAL EPIGRAPH EXAMPLE
%\epigraph{Do not believe everything you see on the internet.}{-- Albert Einstein}


%%%% MUST: if the chapter is a reprint or submitted paper, you must declare it
%% you can use enumerate or itemize environment if you have more than one paper 
%% \mybibexclude{} will exclude this citation from the final bibliography
%% if this paper appears somewhere else then remove \mybibexclude{} command

%\begin{singlespace}         % you can also use `onehalfspace` to relax the spacing
%    This chapter is adapted from the following article with permission from <publisher>
    
%    \fullcite{einstein}. \mybibexclude{einstein}
%\end{singlespace} 


%%%% remove the following and add your chapter text here
\section{Introduction}
The growing demand for technical skills has led many learners to open-access courseware platforms like Coursera and edX. While these platforms offer a wealth of quality educational content, they also present challenges, such as navigating a vast selection of courses, identifying those that align with individual goals, and understanding prerequisite requirements. 

This leads to choice overload, where too many options and insufficient guidance result in frustration and poor learning outcomes. Existing recommendation systems rely on vanilla algorithms that prioritize popular courses or basic keyword matching, and fail to capture the unique needs and diverse backgrounds of learners. Recent advancements in AI, particularly through Large Language Models (LLMs) and Retrieval-Augmented Generation (RAG), offer new opportunities to address these challenges. By providing more dynamic and context-aware recommendations, these models can create personalized experiences. 

This chapter explores the obstacles faced by self-learners, the limitations of existing recommendation systems, and how LLMs can reshape personalized learning. It also identifies the key features of an effective system and the remaining gaps in current solutions.

\section{AI is Changing Every Field} 

Advancements in Artificial Intelligence (AI) are transforming a wide range of industries \cite{nature:2023} and driving the demand for technical skills and Computer Science knowledge \cite{tamayo:2023}. From healthcare \cite{saxena:2024} to education \cite{luckin:2016} and creative industries \cite{almamari:2024}, AI is enabling new innovations that require expertise in areas such as Data Science, Machine Learning, and Software Development. In sectors like healthcare, AI is being used to analyze large datasets, improve diagnostic accuracy \cite{lenharo:2023}, and enhance patient care \cite{ranjbar:2024}. Similarly, in finance, AI-powered algorithms are automating fraud detection and risk assessment \cite{shabir:2024}. These developments demonstrate the expanding influence of AI and highlight the increasing need for professionals with a strong technical foundation.

\subsection{There is a growing demand for technical skills}

As AI and technology continues to shape various fields, the demand for individuals with Computer Science knowledge has surged. According to recent studies, job postings requiring computer science and AI-related skills have increased significantly, emphasizing the rising value of these competencies in the job market \cite{bls:2024} \cite{wef:2020}. 

More professionals and learners are recognizing the value of technical expertise, both to remain competitive in their current roles and to explore new career paths \cite{abe:2020}. Moreover, a growing number of students and early-career professionals are seeking knowledge in these areas \cite{anderson:2023}. This influx of learners from diverse backgrounds creates new challenges in identifying educational resources that are both relevant and appropriate for their various needs.

\subsection{Traditional education is not enough }
However, traditional educational programs are often not well-suited to meet this growing demand \cite{shell:2018}. While university degrees offer comprehensive coursework, they are typically expensive and inflexible, making them inaccessible to many learners. These programs also struggle to meet the needs of continuing learners—those already in the workforce who wish to update their skills \cite{laal:2012}. High tuition costs and rigid schedules limit access to these traditional pathways, leaving many learners seeking alternative ways to gain the necessary expertise \cite{vargas:2023}.

\subsection{MOOCs}
For these reasons, an increasing number of learners are turning to more flexible and affordable options, such as open-access courseware and Massive Open Online Courses (MOOCs) \cite{harish:2013} \cite{papmpouri:2021}. The rise of Massive Open Online Courses marked a significant shift in the accessibility of education. MOOCs were initially celebrated for their potential to democratize education, offering a plethora of high-quality courses from leading universities to anyone with internet connection. 

In 2009, a meta-analysis by the U.S. Department of Education examined evidence-based practices in online learning and found that students in online learning conditions performed better on average than students receiving in-person instruction \cite{means:2009}. However, as the popularity of these platforms grew, studies revealed patterns of enrollment and attrition that highlighted several challenges. A preliminary case study on Australian open-access online education found that while MOOCs attracted a large and diverse group of enrollees, many learners struggled to stay engaged \cite{greenland:2014}. High attrition rates and low completion rates became characteristic of MOOCs, both emphasizing their capability for increasing access to education and raising concerns about their effectiveness in achieving long-term learning outcomes \cite{clow:2013}. 

\section{What a system/solution should include}

Synthesize everything 
- personalized and goal oriented
- take backgrounds into consideration + prerequisite stuff 

\chapter{Literature Review} \label{chap:chap-2}

%%%% OPTIONAL EPIGRAPH EXAMPLE
%\epigraph{Do not believe everything you see on the internet.}{-- Albert Einstein}


%%%% MUST: if the chapter is a reprint or submitted paper, you must declare it
%% you can use enumerate or itemize environment if you have more than one paper 
%% \mybibexclude{} will exclude this citation from the final bibliography
%% if this paper appears somewhere else then remove \mybibexclude{} command

%\begin{singlespace}         % you can also use `onehalfspace` to relax the spacing
%    This chapter is adapted from the following article with permission from <publisher>
    
%    \fullcite{einstein}. \mybibexclude{einstein}
%\end{singlespace} 


%%%% remove the following and add your chapter text here
\section{Introduction}
The growing demand for technical skills has led many learners to open-access courseware platforms like Coursera and edX. While these platforms offer a wealth of quality educational content, they also present challenges, such as navigating a vast selection of courses, identifying those that align with individual goals, and understanding prerequisite requirements. 

This leads to choice overload, where too many options and insufficient guidance result in frustration and poor learning outcomes. Existing recommendation systems rely on vanilla algorithms that prioritize popular courses or simple keyword matching, neglecting the unique needs and diverse backgrounds of learners. 

Recent advancements in AI, particularly through Large Language Models (LLMs) and Retrieval-Augmented Generation (RAG), offer new opportunities to address these challenges. By providing more dynamic and context-aware recommendations, these models can create personalized experiences. However, implementing these systems involves balancing personalization, accuracy, and ethical considerations like privacy.

This chapter explores the issues faced by self-learners, the limitations of existing recommendation systems, and how LLMs can reshape personalized learning. It also identifies the key features of an effective system and the remaining gaps in current solutions.

\section{Problems faced by self learners} 

\subsection{Importance of considering backgrounds in education} 

\subsection{AI is changing every field} 

\section{Existing solutions for self learners and their limitations} 

\subsection{History of MOOCs and Open-Access Education}

This doesn't work bc they use outdated rec systems 

\section{Brief history of recommendation systems}

Choice overload but MOOCs don't do that 

\section{How LLMs are changing recommendation systems}

(what they might offer to solve problems)

\section{What a system/solution should include}

Synthesize everything 
- personalized and goal oriented
- take backgrounds into consideration + prerequisite stuff 

\chapter{Gaita Overview} \label{chap:chap-3}


% epigraph after chapter heading
%\epigraph{Since it is written in \LaTeX, it must be true.}{-- Isaac Newton}


%%%% MUST: add the citation for the chapter if it is a reprint

\section{Overview of Design and Purpose}

\subsection{Goals and Objectives}
We introduce Gaita (Generative AI Teaching Assistant) with the primary goal of providing personalized learning pathways out of open access courses for Computer Science learners by leveraging advanced recommendation techniques. Gaita aims to address the challenge of choice overload by offering tailored recommendations that consider learners' background, prior knowledge, and personal goals. By providing more relevant course suggestions, Gaita seeks to enhance learner engagement, improve completion rates, and support users to succeed in their learning journey. Additionally, Gaita supports diverse learners, including those from other fields, by recommending Computer Science courses that are relevant to their specific area of study, ensuring the learning path aligns with their unique goals and background.

\subsection{How users interact with Gaita to generate recommendations}

Users interact with Gaita through an easy-to-use chatbot interface, where they provide relevant information about their background and learning goals. Upon entering the platform, learners are prompted share details about their current knowledge, experience, and learning objectives. 

Gaita responds with a list of relevant courses from its RAG database of open-access CS courses, along with descriptions of how they align with the user’s prompt and relevant follow-up questions to assess the learner's experience with prerequisites for the recommended courses, which is where the iterative prompting comes into play. Based on the user's responses, Gaita suggests appropriate prerequisite courses to bridge any knowledge gaps. 

The system iteratively refines its suggestions based on the learner's feedback, which ensures that the learning path remains relevant and aligned with the learner's evolving needs.

\subsection{How learners' background and goals influence their learning path}

The input provided by learners—such as their background, previous experience, and specific learning goals—directly shapes the learning paths generated by Gaita. For example, learners with a background in a different field or those transitioning into Computer Science will receive recommendations for foundational courses that address gaps in their knowledge. Meanwhile, learners with prior experience in certain areas will be guided toward more advanced or specialized content. By considering these inputs, Gaita ensures that the recommendations are not only relevant but also adaptable to the learner's unique circumstances.

\section{Justification for Components} 
Justify components chosen; why and why not (reviews, these 2 sites) 

%\blindtext\footnote{Hello, this is the first footnote with no indentation and single-spaced text. The spacing between two footnotes is also single-spaced.}

\section{Data Sources}
\chapter{Conclusion} \label{chap:chap-7}


%% remove the following and add your chapter text here

\section{Key Contributions}
This work introduces Gaita, a novel personalized learning system that leverages conversational AI and Retrieval-Augmented Generation (RAG) to deliver dynamic, context-aware course recommendations. Unlike traditional search algorithms that return results based solely on relevance or other RAG models that aim to prevent hallucination, Gaita harnesses the interpretive power of Large Language Models (LLMs) to offer personalized learning paths that actively engage with the learner’s unique background and aspirations. A key contribution is Gaita’s iterative prompting mechanism, which tailors follow-up questions to assess prerequisite knowledge. While iterative prompting has been explored in various contexts, Gaita is, to our knowledge, the first system to implement it specifically for bridging knowledge gaps and recommending prerequisite courses within personalized learning pathways. This dynamic approach addresses a critical gap in traditional recommendation systems, which often fail to consider the evolving needs of users. 

Moreover, Gaita’s unique ability to take in natural language input is among the first implementations for open courseware, allowing for more nuanced user understanding and an improved user experience. This conversational approach allows Gaita to better interpret user needs and aspirations, offering more personalized and relevant course recommendations than traditional search systems.

Furthermore, Gaita’s modular design allows for easy integration with existing open courseware platforms, such as Coursera and MIT OpenCourseWare, as well as individual university course catalogs. This flexibility expands the range of resources available for personalized recommendations, positioning Gaita as a scalable solution that can cater to learners, ranging from university students to lifelong learners, by offering tailored educational paths across different platforms. Ultimately, this work lays the groundwork for further advancements in personalized learning systems, by incorporating conversational AI and RAG to offer an innovative approach to helping learners navigate complex educational landscapes.

\section{Impact and Implications}


The development of Gaita represents a significant step forward in the application of Artificial Intelligence (AI) and Information Retrieval (IR) technologies in education. By leveraging conversational AI, Gaita transforms how learners interact with the vast array of available educational resources, by tailoring their learning journeys to their unique backgrounds and goals. This approach not only enhances the learning experience for individual users but also has broader implications for the future of EdTech and personalized learning. Gaita’s use of natural language processing enables conversational interactions, which helps bridge the gap between traditional search systems and the personalized, adaptive systems learners increasingly expect.

In the broader context of EdTech, Gaita’s integration with open courseware platforms supports the democratization of education by helping learners from diverse backgrounds, including nontraditional and lifelong learners, and those underserved by conventional systems. By creating personalized learning pathways, Gaita engages users and ensures their learning experience is continuously adapted and refined. This approach showcases the potential of AI-driven systems to not only recommend relevant content but also interact with learners, supporting their pursuit of knowledge and enhancing the overall educational experience.

Ultimately, Gaita demonstrates the potential for Artificial Intelligence and Information Retrieval technologies for creating more accessible, inclusive, and effective education systems. As these technologies continue to evolve, they have the potential to transform the future of education by personalizing learning experiences at scale, empowering learners from diverse backgrounds, and ensuring that education is tailored to the unique needs, backgrounds, and aspirations of every individual.
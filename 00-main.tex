%%%%%%%%%%%%%%%%%%%%%%%%%%%%%%%%%%%%%%%%%%%%%%%%%%%%%%%%%%%%%%%%%%%%%%%%%%%%%%%%
%%%%%%%%%%%%%%%%%%%%%%%%%%%%%%%%%%%%%%%%%%%%%%%%%%%%%%%%%%%%%%%%%%%%%%%%%%%%%%%%
%%                          AUTHOR: BIBEKANANDA DATTA                         %%
%%                               (C) MARCH 2024                               %%
%%                      PhD STUDENT, MECHANICAL ENGINEERING                   %%
%%                           JOHNS HOPKINS UNIVERSITY                         %%
%%%%%%%%%%%%%%%%%%%%%%%%%%%%%%%%%%%%%%%%%%%%%%%%%%%%%%%%%%%%%%%%%%%%%%%%%%%%%%%%
%%%%%%%%%%%%%%%%%%%%%%%%%%%%%%%%%%%%%%%%%%%%%%%%%%%%%%%%%%%%%%%%%%%%%%%%%%%%%%%%

%%%%%%%%%%%%%%%%%%%%%%%%%%%%%%%%%%%%%%%%%%%%%%%%%%%%%%%%%%%%%%%%%%%%%%%%%%%%%%%%
%%             PLEASE CHECK THE README.MD FILE BEFORE YOU PROCEED             %%
%%   GitHub: https://github.com/bibekananda-datta/JHU-Dissertation-Template   %%
%%              it may be conveneint to read this file on GitHub              %%
%%%%%%%%%%%%%%%%%%%%%%%%%%%%%%%%%%%%%%%%%%%%%%%%%%%%%%%%%%%%%%%%%%%%%%%%%%%%%%%%


%%%%%%%%%%%%%%%%%%%%%%%%%%%%%%%%%%%%%%%%%%%%%%%%%%%%%%%%%%%%%%%%%%%%%%%%%%%%%%%%

% This is an unofficial dissertation template for Johns Hopkins University.
% As of March 04, 2024, the template follows the dissertation formatting 
% requirements provided by the  Johns Hopkins University Sheridan Library.
% It is the user's responsibility to ensure that the current requirements are 
% followed: https://www.library.jhu.edu/library-services/electronic-theses-dissertations/formatting-requirements/

%%%%%%%%%%%%%%%%%%%%%%%%%%%%%%%%%%%%%%%%%%%%%%%%%%%%%%%%%%%%%%%%%%%%%%%%%%%%%%%%


%%%%%%%%%%%%%%%%%%%%%%%%%%%%%% VERSION HISTORY %%%%%%%%%%%%%%%%%%%%%%%%%%%%%%%%%

% The report-class-based template was created by R. Jacob Vogelstein in May 2007
% Updated by Noah J. Cowan on March 01, 2010
% Updated by Brian D. Weitzner on April 29, 2014 
% Updated by John Muschelli on January 29, 2016 
% Updated by Leonardo Collado Torres on April 13, 2016 
% Updated by John Clayton in December 2019
% Last Updated by Bibekananda Datta on March 04, 2024

%%% CURRENT VERSION INCLUDES:
% Reorganization of the necessary packages (you may need more) and their settings.
% Modularized all the settings, formatting, and macros. You can add more as needed.
% Added specific settings for adding epigraphs before or after the chapter headings.
% Added settings to customize white space before chapter headings.
% Added header option for each chapter using fancyhdr package. 
% Added settings for including algorithm (algorithm2e) and codes (listing).
% Updated the appearance of table of contents, list of figures, and list of tables.

%%%%%%%%%%%%%%%%%%%%%%%%%%%%%% VERSION HISTORY %%%%%%%%%%%%%%%%%%%%%%%%%%%%%%%%%




%%%%%%%%%%%%%%%%%%%%%%%%%%%%%%%%%%%%%%%%%%%%%%%%%%%%%%%%%%%%%%%%%%%%%%%%%%%%%%%%
%%%%%%%%%%%%%%%%%%%%%%%%%%%%%%%%%%%%%%%%%%%%%%%%%%%%%%%%%%%%%%%%%%%%%%%%%%%%%%%%

%% if possible, please make your formatting changes here through the variables 
%% go through all the variables and understand what role they play in formatting

%%%%%%%%%%%%%%%%%%%%%% LIST OF VARIABLES FOR FORMATTING %%%%%%%%%%%%%%%%%%%%%%%%

\def\FontPackage{lmodern}                   % latin modern font (you can also change it to times)
\def\BibFileName{thesis.bib}                % name of BibLaTeX file with all the bibliography
\def\FigurePath{figures}                    % subdirectory for the figure files


\def\NoSectionLevel{3}                      % 3 levels for sections ... to subsubsection
\def\TocIndent{0}                           % indentation in the list of figs and tables
\def\NoTocLevel{3}                          % no of levels showed in the table of contents
%% 3 levels mean section to subsubsection.. decrease if you want less to showup in TOC

\def\MainTextSpacing{\doublespacing}        % double spacing in main text (JH library requirement)
\def\TOCTextSpacing{\onehalfspacing}        % one-half spacing for TOC texts
\def\BibTextSpacing{\singlespacing}         % single spacing for bibliography
%% JH library does not specify spacing for TOC and bibliographic references


\def\ChapterTopSpace{-48}                   % white space on top of the chapter heading (unit: pt)
\def\ChapterToTitle{-12}                    % space between chapter to title (unit: pt)
\def\TitleToText{18}                        % space between the chapter title to the following text (unit: pt)


% font format for chapter heading and title
\def\ChapterFont{\singlespacing \Large \bfseries}
\def\SectionFont{\large\bfseries}           % section heading font format
\def\SubsectionFont{\normalsize\bfseries}   % subsection heading font format
\def\SubsubsectionFont{\normalsize\itshape} % subsubsection heading font format
\def\CaptionFontSize{small}                 % caption font size
\def\CaptionFontType{bf}                    % boldface label for captions
\def\CaptionSeparator{colon}                % separates caption heading from text. can use 'period' as well
\def\CodeFont{\footnotesize\ttfamily}       % font for including codes using listing


%% if you have a longer quote, you may have to change it.
\def\QuoteWidth{0.65\textwidth}             % width of quote in epigraph


%% if this seems too widespread for you, try changing it locally using
%% \begin{group} ... \renewcommand{\arraystretch} ... \end{group} commands
\def\GlobalTableSpacing{1.5}                % global spacing parameter for table


\def\ParagraphSpacing{\baselineskip}        % spacing between paragraph
\def\ParagraphIndent{0}                     % indentation at the beginning of the paragraph
\def\FullCiteSpacing{1.25}                  % spacing in a fullcite item
\def\BibItemSpacing{\baselineskip}          % spacing between bibliographic items in reference
\def\FootnoteSpacing{0.75\baselineskip}     % spacing between footnotes
\def\CaptionSpacing{0}                      % spacing between the figure and the caption (unit: pt)

%% pdflatex compression settings to generate compressed PDF
\pdfcompresslevel=9
\pdfminorversion=5
\pdfobjcompresslevel=2

%%%%%%%%%%%%%%%%%%%% END LIST OF VARIABLES FOR FORMATTING %%%%%%%%%%%%%%%%%%%%%%





%%%%%%%%%%%%%%%%%%%%%%%%%%%%%%%%%%%%%%%%%%%%%%%%%%%%%%%%%%%%%%%%%%%%%%%%%%%%%%%

%% add packages as needed but sometimes the order of the packages matter.
%% I primarily tried to load the packages in alphabetical order unless or 
%% based on their functionalities (all math/ table packages) unless there 
%% is an issue with dependency, then I included packages in that order.
%% you may get warning/ error for the order in which packages are included
%% you may have to change the options in biblatex package for bibliography

%%%%%%%%%%%%%%%%%%%%%%%%%% LaTeX CLASS AND PACKAGES %%%%%%%%%%%%%%%%%%%%%%%%%%%

\documentclass[12pt]{report}                % report document class w/ 12 pt font

\usepackage[utf8]{inputenc}	                % for encoding input character
\usepackage[pagewise,mathlines]{lineno}     % linenumbers for drafting


%% math packages
\usepackage{amsfonts,amssymb,amsmath,amsthm,autobreak,cancel,dsfont,mathtools,mathbbol,mathrsfs,siunitx,upgreek}

\usepackage[ruled]{algorithm2e}             % to manage algorithm environment
\usepackage[titletoc]{appendix}             % to manage appendix chapters
\usepackage[american]{babel}                % for different language typography

%% bibliographic package (make sure your bib file is in BibLaTeX format)
%% use Zotero or some other reference manager to generate the BibLaTeX file
%% change the style or other options if you need to
\usepackage[backend=biber, style=nature, maxnames=9, date=year, isbn=false, url=false, doi=true]{biblatex}
% \usepackage[backend=biber, style=apa, isbn=false, url=false, doi=true]{biblatex}

\usepackage{blindtext}                      % to generate random filler texts
\usepackage{calc}                           % to set arithmetic arguments for spacing
\usepackage{caption}                        % to manage captions
\usepackage{color}                          % color related packages
\usepackage{csquotes,epigraph,varwidth}     % for managing quotes
\usepackage{enumitem}                       % to manage list environment
\usepackage{float}                          % to manage floating environment
\usepackage[T1]{fontenc}                    % for font encoding
\usepackage[bottom]{footmisc}               % footnote environment management
\usepackage{graphicx,wrapfig}               % to manage images
\usepackage{geometry}                       % to manage margins and others
\usepackage{fancyhdr}                       % for header/ footer settings
\usepackage[dvipsnames]{xcolor}
\usepackage[a-1b]{pdfx}                     % to generate PDF/A file (before hyperref)
\usepackage[pdfa]{hyperref}                 % for hyperlinks
\usepackage[all]{hypcap}                    % for captions on the side of figures
\usepackage{ifthen}                         % if-then statement in algorithm
\usepackage{lscape}                         % landscape mode
\usepackage{listings}                       % to include codes

%% table related packages
\usepackage{booktabs,longtable,makecell,multicol,multirow,tabularx,xltabular}


\usepackage{setspace}                       % sets space between lines
\usepackage{seqsplit}                       % splits long character sequence
\usepackage[rightcaption]{sidecap}          % for sideway captions
\usepackage[titles]{tocloft}                % to manage table of contents
\usepackage{parskip}
\usepackage{textcomp}                       % text companion fonts in TS1
\usepackage{titlesec}                       % managing different titles
\usepackage{tikz}                           % drawing related package
\usepackage{subcaption}                     % individual panel and caption

%% add more packages and/or change options of the packages as needed

%%%%%%%%%%%%%%%%%%%%%%%%% END LaTeX CLASS AND PACKAGES %%%%%%%%%%%%%%%%%%%%%%%%%






%%%%%%%%%%%%%%%%%%%%%%%%%%%%%%%%%%%%%%%%%%%%%%%%%%%%%%%%%%%%%%%%%%%%%%%%%%%%%%%

%% if possible, make all of your changes through the defined variables above.
%% you can make most common changes there before tweaking following settings.
%% if you are really specific about some formatting and can not change them
%% by changing defined variables in the beginning, only then proceed to the 
%% next sections which includes loading proper package options, redefining 
%% different environments, document formatting, settings for special packages 
%% (epigraph, algorithm, listings, etc). be cautious before making any changes.

%%%%%%%%%%%%%%%%%%%%%%%%%%%%%% PACKAGE OPTIONS %%%%%%%%%%%%%%%%%%%%%%%%%%%%%%%%

%% add all the images to that folder for cleaner file management
%% you can add images in chapter-wise PDF format (my preference).
\graphicspath{{\FigurePath/}}


%% this file has to be in BibLaTeX format. Use Zotero or some other citation manager to generate the .bib file in BibLaTeX format.
\addbibresource{\BibFileName}


%% margin settings required by JH library using geometry package
%% if you have a long chapter title you may need to customize the header settings here
\geometry{letterpaper, left=1.5in, right=1.0in, top=1.0in, bottom=1.0in, includehead, headheight=30pt, headsep=10pt, includefoot, heightrounded}


%% settings for the hyperref package
\hypersetup{linktocpage, unicode, colorlinks=true, citecolor=blue, filecolor=blue, linkcolor=blue, urlcolor=blue}
% add 'linktoc=all' option to the above list for making the items in TOC as clickable links
\urlstyle{rm}           % removes default \texttt style for hyperlinks


%% settings for caption package (customize or add more as needed)
\captionsetup{belowskip=\CaptionSpacing pt, font=\CaptionFontSize, labelfont=\CaptionFontType, labelsep=\CaptionSeparator, hypcap=true} 


%% settings for listing package to include code
\lstset{basicstyle=\CodeFont,columns=flexible,breaklines=true}


%% settings for TikZ library (you can add more settings here)
\usetikzlibrary{positioning,shapes,arrows}


%% may need to define more unicode characters if they appear in your thesis
\DeclareUnicodeCharacter{2212}{-}           % defining unicode character '-'


%%%%%%%%%%%%%%%%%%%%%%%%%%%% END PACKAGE OPTIONS %%%%%%%%%%%%%%%%%%%%%%%%%%%%%%%


%%%%%%%%%%%%%%%%%%%%%%%% REDEFINITION OF ENVIRONMENTS %%%%%%%%%%%%%%%%%%%%%%%%%%

%%%% UNNUMBERED CHAPTERS, SECTION, and SUBSECTION COMMAND for ADDING to TOC
% removes the 'Chapter #' title while keeping it listed in the TOC
\newcommand\chap[1]{%
  \chapter*{#1}%
  \markboth{#1}{}
  \addcontentsline{toc}{chapter}{#1}}
  
% removes the 'Section #' title while keeping it listed in the TOC
\newcommand\sect[1]{%
  \section*{#1}%
  \addcontentsline{toc}{section}{#1}}
  
% Removes the 'Subsection #' title while keeping it listed in the TOC
\newcommand\subsect[1]{%
  \subsection*{#1}%
  \addcontentsline{toc}{subsection}{#1}}

% Removes the 'Subsubsection #' title while keeping it listed in the TOC
\newcommand\subsubsect[1]{%
  \subsubsection*{#1}%
  \addcontentsline{toc}{subsubsection}{#1}}

%%%%%%%%%%%%%%%%%%%%%%% END REDEFINITION OF ENVIRONMENTS %%%%%%%%%%%%%%%%%%%%%%%


%%%%%%%%%%%%%%%%%%%%%%%%%%%%% DOCUMENT FORMATTING %%%%%%%%%%%%%%%%%%%%%%%%%%%%%%

%%%% choice a font form (or add something else) for your thesis
\usepackage{\FontPackage}

%%%% TOC shows (chapter to subsection) in the list
\setcounter{tocdepth}{\NoTocLevel}
\setcounter{secnumdepth}{\NoSectionLevel}       % section to ... subsubsection ...


\setlength{\cftfigindent}{\TocIndent pt}        % indentation from figures in lof
\setlength{\cfttabindent}{\TocIndent pt}        % indentation from tables in lot


%%%% dots for chapters too
\renewcommand{\cftchapleader}{\cftdotfill{\cftdotsep}}


% tweak to TOC to add 'chapter' to the chapter name instead of a number only
% set the width of the box based on the longest label name
\renewcommand{\cftchappresnum}{\chaptername\space}
\setlength{\cftchapnumwidth}{\widthof{\textbf{Appendix~999~}}}


% tweak to TOC to add 'Figure' to the figure caption listing
% to change the distance to the start of the figure title
\renewcommand{\cftfigpresnum}{\bfseries Figure }
\setlength{\cftfignumwidth}{\widthof{\textbf{Figure~99.999~}}}


% tweak to TOC to add 'Table' to the Table caption listing
% to change the distance to the start of the figure title
\renewcommand{\cfttabpresnum}{\bfseries Table }
\setlength{\cfttabnumwidth}{\widthof{\textbf{Table~99.100~}}}



%%%% chapter # and title settings (with white space)
%% if you use an epigraph after the chapter title then perhaps reduce the first \vspace* from 0 pt to -(some_value)
\makeatletter
\def\@makechapterhead#1{%
  \vspace*{\ChapterTopSpace \p@}   % white space before the chapter #
  {\parindent \z@ \raggedright \normalfont
    \ifnum \c@secnumdepth >\m@ne
        \ChapterFont \@chapapp\enskip \thechapter 
        \\ \vspace{\ChapterToTitle \p@}   % space between chapter # and title
    \fi
        \interlinepenalty\@M
        \ChapterFont #1\par\nobreak
    \vskip \TitleToText \p@         % space between the chapter title and the following text
  }}
\makeatother


%%%% settings for unnumbered chapters
% if you use an epigraph after the chapter title then perhaps reduce the first \vspace* from 0 pt to - (some_value)
\makeatletter
\def\@makeschapterhead#1{%
  \vspace*{\ChapterTopSpace \p@} % white space before to the chapter title
  {\parindent \z@ \raggedright
    \normalfont
    \interlinepenalty\@M
    \ChapterFont  #1\par\nobreak
    \vskip \TitleToText \p@     % space between the chapter title and the following text
  }}
\makeatother


%%%% using titlesec package for sections, subsection, ... heading format
\titleformat*{\section}{\SectionFont}
\titleformat*{\subsection}{\SubsectionFont}
\titleformat*{\subsubsection}{\SubsubsectionFont}


%%%% settings for paragraph (and not title) spacing, roughly speaking
\renewcommand{\arraystretch}{\GlobalTableSpacing}   % spacing inside table
\setlength{\parskip}{\ParagraphSpacing}             % paragraph skip
\setlength{\parindent}{\ParagraphIndent pt}         % paragraph indentation
\setlength{\bibitemsep}{\BibItemSpacing}            % bib item separation 
\setlength{\footnotesep}{\FootnoteSpacing}          % separation between footnote


%%%% bibliography package settings
\DeclareFieldFormat{titlecase}{\MakeSentenceCase*{#1}}
\AtBeginBibliography{\urlstyle{rm}}
\DeclareBibliographyCategory{fullcited}
\newcommand{\mybibexclude}[1]{\addtocategory{fullcited}{#1}}

%%%%%%%%%%%%%%%%%%%%%%%%%%% END DOCUMENT FORMATTING %%%%%%%%%%%%%%%%%%%%%%%%%%%



%%%%%%%%%%%%%%%%%%%%%%%%%%%%%%%%%%%%%%%%%%%%%%%%%%%%%%%%%%%%%%%%%%%%%%%%%%%%%%%

%% following sections are optional and will be required for using specific packages

%% if you plan on using quotes, you may need specialized epigraph settings.
%% hopefully the following settings will work for you if the quote is small.
%% epigraph examples given in chapters 2 and 3 for relatively small quotes.

%%%%%%%%%%%%%%%%%%%%%%%%%%%%% EPIGRAPH SETTINGS %%%%%%%%%%%%%%%%%%%%%%%%%%%%%%%

%%%% Following settings allow the epigraph and the underline 
%% settings for arbitrary epigraph length
\renewcommand{\epigraphflush}{flushright}
\renewcommand{\epigraphsize}{\small}
\setlength{\epigraphwidth}{\QuoteWidth}
\renewcommand{\textflush}{flushright}
\renewcommand{\sourceflush}{flushright}
% A useful addition
\newcommand{\epitextfont}{\itshape}
\newcommand{\episourcefont}{\scshape}

\makeatletter
\newsavebox{\epi@textbox}
\newsavebox{\epi@sourcebox}
\newlength\epi@finalwidth
\renewcommand{\epigraph}[2]{%
  \vspace{\beforeepigraphskip}
  {\epigraphsize\begin{\epigraphflush}
   \epi@finalwidth=\z@
   \sbox\epi@textbox{%
     \varwidth{\epigraphwidth}
     \begin{\textflush}\epitextfont#1\end{\textflush}
     \endvarwidth
   }%
   \epi@finalwidth=\wd\epi@textbox
   \sbox\epi@sourcebox{%
     \varwidth{\epigraphwidth}
     \begin{\sourceflush}\episourcefont#2\end{\sourceflush}%
     \endvarwidth
   }%
   \ifdim\wd\epi@sourcebox>\epi@finalwidth 
     \epi@finalwidth=\wd\epi@sourcebox
   \fi
   \leavevmode\vbox{
     \hb@xt@\epi@finalwidth{\hfil\box\epi@textbox}
     \vskip 1ex         % gap between quote and rule
     \hrule height \epigraphrule
     \vskip 1ex         % gap between rule and author
     \hb@xt@\epi@finalwidth{\hfil\box\epi@sourcebox}
   }%
   \end{\epigraphflush}
   \vspace{\afterepigraphskip}}}
\makeatother

%%%%%%%%%%%%%%%%%%%%%%%%%%%%% EPIGRAPH SETTINGS %%%%%%%%%%%%%%%%%%%%%%%%%%%%%%%



%%%%%%%%%%%%%%%%%%%%%%%%%%%%%%%%%%%%%%%%%%%%%%%%%%%%%%%%%%%%%%%%%%%%%%%%%%%%%%%

%% if you plan on adding algorithms and codes in your thesis, the these settings
%%  may be helpful. you can tweak them based on your need and preferences.

%%%%%%%%%%%%%%%%%%%%%% ALGORITHM AND LISTING SETTINGS %%%%%%%%%%%%%%%%%%%%%%%%%

%% settings for algorithm2e package
\renewcommand{\algorithmcfname}{Procedure}
\SetKwFor{While}{while}{}{end while}%
\SetArgSty{textnormal}
\newcommand\mycommfont[1]{\footnotesize\ttfamily\textcolor{blue}{#1}}
\SetCommentSty{mycommfont}


%% listing package definition (to add code in the document)
\lstdefinestyle{terminal}{columns=fullflexible,
keepspaces=true,
breaklines=true,
basicstyle={\footnotesize\fontfamily{fvm}\fontseries{m}\selectfont},
keywordstyle={\footnotesize\fontfamily{fvm}\fontseries{b}\selectfont},
commentstyle={\color{comments}\small\fontfamily{fvm}\itshape\selectfont},
frame=single,
xleftmargin=0in,
backgroundcolor=\color{lightgray!50},
belowcaptionskip=10pt,
aboveskip=0.5cm}
\lstset{style=terminal,float=h,language=bash}

%%%%%%%%%%%%%%%%%%%% END ALGORITHM AND LISTING SETTINGS %%%%%%%%%%%%%%%%%%%%%%%



%%%%%%%%%%%%%%%%%%%%%%%%%%%%%%%%%%%%%%%%%%%%%%%%%%%%%%%%%%%%%%%%%%%%%%%%%%%%%%%

%% add all your custom math settings and macros in the following section.
%% this is where LaTeX supremacy becomes a thing. you can customize a lot.

%%%%%%%%%%%%%%%%%%%%%%% MATH SETTINGS AND MACROS %%%%%%%%%%%%%%%%%%%%%%%%%%%%%%

\allowdisplaybreaks[1]
\setcounter{MaxMatrixCols}{20}          % no of maximum columns in matrix
\numberwithin{equation}{chapter}        % eqn no with chapter prefix
\newcommand{\dC}{$^{\circ}$C}           % degree celcius symbol
\newcommand{\vect}[1]{\mathbf{#1}}      % boldface for vectors and tensors
\DeclareMathOperator{\tr}{tr}           % trace of a matrix
\DeclareMathOperator{\divg}{div}        % divergence of vector and tensor
\DeclareMathOperator{\grad}{grad}       % gradient of vector and tensor

%% these are just some examples; add more macros for your custom symbols

%%%%%%%%%%%%%%%%%%%%% END MATH SETTINGS AND MACROS %%%%%%%%%%%%%%%%%%%%%%%%%%%%



%%%%%%%%%%%%%%%%%%%%%%%%%%%%%%%%%%%%%%%%%%%%%%%%%%%%%%%%%%%%%%%%%%%%%%%%%%%%%%%

%% add all your non-mathematical macros and other random settings in here.

%%%%%%%%%%%%%%%%%%%%%%%%%%%%%% OTHER MACROS %%%%%%%%%%%%%%%%%%%%%%%%%%%%%%%%%%%

\newcommand{\COMMENT}{\textcolor{red}}
\newcommand{\ADDCITATION}{\COMMENT{(ADD CITATION)}}

%% you can also add more simple comments here as you need
%% you can use some other packages for more complicated review and comment section

%%%%%%%%%%%%%%%%%%%%%%%%%%%% END OTHER MACROS %%%%%%%%%%%%%%%%%%%%%%%%%%%%%%%%%





%%%%%%%%%%%%%%%%%%%%%%%%% DOCUMENT BEGINS HERE %%%%%%%%%%%%%%%%%%%%%%%%%%

\begin{document}


%%%%%%%%%%%%%%%%%%%%%%%%%%%% FRONT MATTER %%%%%%%%%%%%%%%%%%%%%%%%%%%%%%%%

%%%% JHU Dissertation title page (if you are not sure, do not change the formatting)

\begin{titlepage}

% centered single-spaced title page w/ no page numbering
\begin{center}
\singlespacing  \thispagestyle{empty}       

%%%% thesis title: 1.5 inches from the top of the page
\ThesisTitle{\LaTeX\ Dissertation Template for Johns Hopkins University}

\vspace{1in}                    % gap between the title and the author: approx. 1 inch

\ThesisAuthor{John Doe}         % author name for the thesis

\vspace{1.5in}                  % gap between the author and statement: 1.5 inches

%%%% for masters change arguments to: {thesis}{masters program name}
\ThesisStatement{dissertation}{Doctor of Philosophy}  
%% gap between statement and location: approx. 0.5 inch
\vspace{0.5in} \\               

\Location \\                    % prints Baltimore, Maryland as the location
\ThesisDate{Month}{YEAR}        % thesis submission month and year (single-spaced)


%%%% optional copyright statement: approx. 2 inches from the bottom of the page
%% year and name are input argument for copyright statement
\ThesisCopyright{YEAR}{John Doe}


\end{center}

\end{titlepage}
                          % include the title page
% \linenumbers                              % may find useful during drafting
% place above comment based on where you want to start line numbering

%% chapter header and page numbering set up
%% header unnumbered chapters (to remove the headers, comment out the following lines)
\pagestyle{fancy}
\fancyhead[R]{}                             % empty top-right header throughout the document
\fancyhead[L]{\nouppercase \leftmark}       % avoiding upper-case in header
\pagenumbering{roman}                       % pagination style: roman numeral
\setcounter{page}{2}                        % page counter starts at roman ii
\MainTextSpacing                            % double spacing for the contents


%% add abstract, dedication, and acknowledgment (and any other front matter)
\chap{Abstract} 

%% your abstract goes here (add your abstract)
\Blindtext[3]


%%%%  committee members go here (do not go to a new page; 
%%% it should be right after the abstract)
\begin{singlespace}

\subsection*{Primary reader and thesis advisor:}

Dr. Chuck Darwin \\
Professor\\
Department of Biology\\
Johns Hopkins University, Baltimore MD 

\vspace{0.25in}

\subsection*{Secondary readers: }

Dr. Stewart Hawking\\
Professor\\
Department of Biology \\
Johns Hopkins University, Baltimore, MD 

\vspace{0.1in}

Dr.~Jimmy Watson \\
Professor\\
Department of Biology \\
Johns Hopkins University, Baltimore, MD 

%%%% add more readers if you have in your committee 
%%%% using \vspace{0.1in} in between them

\end{singlespace}
\chapter*{~}
\addcontentsline{toc}{chapter}{Dedication}
%% did not use \chap command because the chapter does not have any name

%%%% there is no format for this page. depending on your text, 
%%%% you can adjust the height and style of the text
\begin{center}
\vspace*{2.5in}
    \textit{This thesis is dedicated to ...}
\end{center}

\chap{Acknowledgement}

%%%% there is no specific formatting requirement for this section 
%% default is double-spaced (you can style it as you want)
TODO

Ram Kripa: help with coding and deployment 

Amanda Ferber: help with scraping and database creation

Prof David yarowsky 

Prof Joshua Reiter 

%% single spacing is too cluttered and double spacing is too wide-spread
\TOCTextSpacing                             % one-half spacing for the table of contents
\renewcommand{\contentsname}{Table of Contents}
\tableofcontents
\listoftables
\addcontentsline{toc}{chapter}{List of Tables}
\listoffigures
\addcontentsline{toc}{chapter}{List of Figures}

%%%%%%%%%%%%%%%%%%%%%%%%%%% END FRONT MATTER %%%%%%%%%%%%%%%%%%%%%%%%%%%%





%%%%%%%%%%%%%%%%%%%%%%%%%%%%%% MAIN TEXT %%%%%%%%%%%%%%%%%%%%%%%%%%%%%%%%

\clearpage                                  % flushes the floats and new page
\pagenumbering{arabic}                      % arabic page numbering
\MainTextSpacing                            % restores double spacing in chapter contents

%% full header for the numbered chapters (to remove header comment out the following lines)
\renewcommand{\chaptermark}[1]{\markboth{#1}{#1}}
\fancyhead[L]{\chaptername\ \thechapter. \nouppercase \leftmark}


%% include all the main text chapters
\chapter{Chapter title goes here} \label{chap:chap-1}

% if you want a short header you can use the following command
% \chapter[short-header-name]{chapter-title} \label{chap:chap-1}


% add your chapter text here

\blindtext

\Blindtext[4] \cite{dirac}
\chapter{Chapter title goes here} \label{chap:chap-2}

% if you want a short header you can use the following command
% \chapter[short-header-name]{chapter-title} \label{chap:chap-2}


\epigraph{\enquote{Do not believe everything you see on the internet.}}{-- Albert Einstein}

% if you want to add the quote above the chapter heading then do the following:
% (for this case, you may have to change \ChapterTopSpace in the main file)
% \epigraphhead[0]{whole-above-epigraph-goes-here}


%%%% add the citation for the chapter if it is a submitted or reprint 
%% \mybibexclude{} will exclude this from the final bibliography
%% if this paper appears somewhere else then remove \mybibexclude
This chapter is adapted from the following publication with permission from \emph{publisher}:
\begin{spacing}{\FullCiteSpacing}
    \fullcite{einstein}. \mybibexclude{einstein}
\end{spacing}



% add your chapter text here
\section{Introduction}
\Blindtext[2]


\blindtext

\chapter{Chapter title goes here} \label{chap:chap-3}

% if you want a short header you can use the following command
% \chapter[short-header-name]{chapter-title} \label{chap:chap-3}


% epigraph after chapter heading
\epigraph{\enquote{Since it is written in \LaTeX, it must be true.}}{-- Isaac Newton}


% add citation for the chapter if it is a reprint


% remove the following and add your chapter text
\section{Introduction}
\blindtext 


\section{Current approach}
\blindtext\footnote{Hello, this is the first footnote.}

\subsection{Hypothesis statement}
\blindtext\footnote{This is the second footnote.}


\subsection{Experimental evidences}
\blindtext


\subsubsection{Data analysis}
\blindtext
\chapter{Chapter title goes here} \label{chap:chap-4}

% if you want a short header you can use the following command
% \chapter[short-header-name]{chapter-title} \label{chap:chap-4}


% add your chapter text here
\section{Introduction}
\blindtext

\begin{figure}[ht]
\begin{center}
    \includegraphics[width=\textwidth, trim={6cm 5cm 6cm 5cm},clip,page=1] {chap2.pdf}
    \caption{Here are some photos of ducks to make you feel happy in tough times.}
    \label{fig:ducks}
\end{center}
\end{figure}


\blindtext[2]
\chapter{Chapter title goes here} \label{chap:chap-5}

% if you want a short header you can use the following command
% \chapter[short-header-name]{chapter-title} \label{chap:chap-5}


% add your chapter text here
\section{Introduction}
\blindtext \parencite{knuth-fa}


\begin{table}[h!]
\centering
\begin{tabular}{c c c c} 
\toprule \toprule
Col1 & Col2 & Col2 & Col3 \\ 
\toprule \toprule
1 & 6 & 87837 & 787 \\ 
2 & 7 & 78 & 5415 \\
3 & 545 & 778 & 7507 \\
4 & 545 & 18744 & 7560 \\
5 & 88 & 788 & 6344 \\ 
\bottomrule
\end{tabular}
\caption{Table to test captions and labels taken from Overleaf.}
\label{table:1}
\end{table}

%% add more chapters as you need

%%%%%%%%%%%%%%%%%%%%%%%%%%%% END MAIN TEXT %%%%%%%%%%%%%%%%%%%%%%%%%%%%%





%%%%%%%%%%%%%%%%%%%%%%%%%%%%% BACK MATTER %%%%%%%%%%%%%%%%%%%%%%%%%%%%%%

%% references
\clearpage
\BibTextSpacing                     % single spacing for the bib items (too widespread otherwise)
% you can also change 'BibItemSeparation' variable to control the spacing
% header unnumbered chapters (to remove the headers, comment out the following line)
\fancyhead[L]{\nouppercase \leftmark}

\chap{Bibliographic references}

\vspace{8pt}                % extra space added to compensate for single-spaced bibliography
\printbibliography[heading=none,notcategory=mypapers]             % include reference chapter


%% appendix chapters
\MainTextSpacing                    % restoring double spacing for the contents in appendices
%% full header for appendix chapters (to remove the headers, comment out the following line)
\fancyhead[L]{\appendixname\ \thechapter. \nouppercase \leftmark}

%% necessary customization for the appendix and headers
\appendix 
\makeatletter
\addtocontents{toc}{\protect\renewcommand\protect\cftchappresnum{\@chapapp\ }}
\makeatother
\renewcommand{\thechapter}{\Alph{chapter}}

\chapter{Some necessary information}

% add your chapter text here
\Blindtext[3]
\chapter{A few more additional information} \label{chap:appendix-b}

%% add your chapter text here
\blindtext

%% add more appendix chapters as you need


%% you can also add biographical sketch/ CV in here 
%% you can write it as other chapters and add the chapter in here
%% if you already have CV as a PDF, add it to the main directory, then use:

% \includepdf[pages=-]{John_Doe_CV.pdf}

%%%%%%%%%%%%%%%%%%%%%%%%%%% END BACK MATTER %%%%%%%%%%%%%%%%%%%%%%%%%%%%



\end{document}

%%%%%%%%%%%%%%%%%%%%%%%%% DOCUMENT ENDS HERE %%%%%%%%%%%%%%%%%%%%%%%%%%%